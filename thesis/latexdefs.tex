\usepackage{xcolor}%for enabling colours in documents
\definecolor{unswred1}{RGB}{238,49,36}
\definecolor{unswred2}{RGB}{197,40,28}
\definecolor{unswred3}{RGB}{158,28,15}
\definecolor{unswred4}{RGB}{122,12,0}
\definecolor{unswblue1}{RGB}{0,174,239}
\definecolor{unswblue2}{RGB}{0,146,200}
\definecolor{unswblue3}{RGB}{0,118,163}
\definecolor{unswblue4}{RGB}{0,91,127}
\definecolor{unswyellow}{RGB}{255,212,0}
\usepackage{pslatex}%for using post-script fonts
\pdfpagewidth=210 true mm%fix for pslatex
\pdfpageheight=297 true mm%fix for pslatex
\usepackage[margin=2.54cm]{geometry}%for setting margins correctly
\usepackage[english]{babel}%for dealing neatly with umlautes
\usepackage{units}%for consistent use of units
\usepackage{url}%for type setting urls
\usepackage{hyperref}%for internal document links
\hypersetup{pdfborder=0 0 0}
\hypersetup{colorlinks=true}
\hypersetup{allcolors=unswred2}
\usepackage{graphicx}%for including graphics
\usepackage{array}%for better table control and commands
\setlength{\extrarowheight}{1pt}
\usepackage{verbatim}%for typesetting code
\usepackage{latexsym}%for a few extra symbols - alternatively use amssymb
\usepackage{sectsty}%for nice colourful headings
\chapterfont{\color{unswred2}}
\sectionfont{\color{unswred2}}
\subsectionfont{\color{unswred2}}
\subsubsectionfont{\color{unswred2}}
\usepackage{tocbibind}%for adding bibliography to toc

\renewcommand{\baselinestretch}{1.5}%


