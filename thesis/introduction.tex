\chapter{Introduction}\label{ch:intro}

Human activities reply on intensive use of energy generated from various source. Primarily, energy can further divided into two groups, renewable and non-renewable. Renewable energy which has drawn more and more attention in recent years, is energy supplied from renewable source. Renewable energy sources including solar, hydro, wind, biomass, tidal wave, ocean thermal and the like. They may be replenished in nature on human timescale, which mean there is little chance that the energy could be depleted. The four major areas which rely on renewable energy are: electricity production, heating/cooling, transportation, and rural energy services. 

Solar power as one of the most important renewable energy is generated when energy from the sun (sunlight). Energy gather from sunlight is converted into electricity or used to heat air, water, or other fluids. There are two mainstream technology to harvest solar energy and put it into good use. The first technology is solar thermal which thermal energy is resulted from concentration of solar radiation. Air, water, or other fluid is widely used as carrier to carry harvested thermal energy which usually can be used directly for heating up space or  generating electricity by means of steam and turbine. Solar thermal is commonly used in residential area to provide hot water for living. Solar thermal electricity, also known as concentrating solar power, is typically designed for large scale power generation. The Ivanpah Solar Electric Generating System is a good example of concentrated solar thermal plant in the Mojave Desert. Solar \gls{PV} converts sunlight directly into electricity using photovoltaic cells by means of photovoltaic effect. \gls{PVS} can be mounted on rooftops, integrated into building designs and vehicles, or scaled up to multi-gigawatt scale power plants. \gls{PVS} can also be used together with concentrating mirrors or lenses for large scale centralized power. It is also not rare to combine solar thermal and \gls{PV} technology into a single system which provide heat as well as electricity at the same time.

With great ambitions to create a  green future for human beings and pressure to reduce production of greenhouse gas, many governments of developed and developing countries encourage people to increase the usage of renewable energy. The installation of \gls{PV} have been increasing within recent decade. It is safe to say, the total number of installation would keep increasing for a long time before human being can find some better solutions to solve the thirst of reliable and clean energy. 

Nowadays, majority of household electrical appliances rely on \gls{AC}, however, solar \gls{PV} is only capable of generating \gls{DC} current or voltage. As a result, in order to make the most the solar power, various types of DC to AC converters are needed to meet different kinds of requirements for different applications. Microinverter, one of the most popular converters which features simplicity and great efficiency for solar applications has become more and more prevalent than ever. Although suffers from slightly higher upfront cost compared with PV strings with centralized inverter solutions. The market calls for cheaper and smaller microinverters in order to help integrate solar panels together with inverter as a single solution which would make building solar \gls{PVS} more convenient and much easier.

The focus of this project is the development of a \gls{SSGPVS} which might be useful in real life. The development may be finished using simulation and no real circuit prototype is required due to limitation on time and budget.  However, in order to make the project as close to real life product development as possible, \gls{HIL} is used which enable the possibility to divide the system into two parts, simulated and actual part. The controller of the system would be implemented using a digital signal controller(DSC) or a microcontroller which is the real hardware in the system and the rest of the systme including \gls{PV} panels, switching elements and grid are simulated using a real time simulator. The simulator which is developed by Opal-RT is able to simulate any amount of time and generate the results using the same amount of time in real life while maintains a small enough time resolution. The closing stage of this project, I have managed to make....



Chapter~\ref{ch:background} explains the background for this document.
Chapter~\ref{ch:style} states the style and submission related requirements
to theses submitted at the school.
Chapter~\ref{ch:content} explains content related requirements to theses and
how to avoid some commonly seen problems.
Chapter~\ref{ch:eval} evaluates the thesis requirements template.  Finally,
Chapter~\ref{ch:conclusion} draws up conclusions and suggests ways to
further improve the thesis requirements template.

